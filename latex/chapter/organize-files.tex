% 根据是否是从 main.tex 开始编译(全部编译)
\ifx\allfiles\undefined
  \documentclass[UTF8,a4paper,10pt]{ctexart}
  \usepackage{../config/mycolor}
  \usepackage{minted}
  \begin{document}
  \title{组织\LaTeX工程}
  \author{fanhongtao@163.com}
  \date{2018 年 6 月 12 日}
  \maketitle
  \pagebreak
  \tableofcontents
  \pagebreak
\else
  \chapter{组织\LaTeX工程}
\fi

% 从这里开始是本文件的内容
\section{配置文件}

可以将会被多个 .tex 使用的内容放在 配置文件 (.sty) 中。然后在 .tex 文件中引用该配置文件。

如:当我在 Visual Studio Code 中编辑时,就喜欢将 PDF 文档设置成黑底白字,
以便和 VSC 的颜色相似,这样分屏显示PDF时,就不会因为色差太大而刺眼。具体方法如下。

1、创建一个名为 mycolor.sty 的文件,其内容为:

\begin{minted}[frame=single]{tex}
  \usepackage{color}
  \definecolor{bgcolor}{rgb}{0.12,0.12,0.12}
  \definecolor{fontcolor}{rgb}{0.95,0.95,0.95}
  \pagecolor{bgcolor}   % set background
  \color{fontcolor}     % set color for font
\end{minted}

2、在 .tex 文件中引用 color,并设置背景、字体颜色

\begin{minted}[frame=single]{tex}
  \documentclass[UTF8,a4paper]{ctexart}
  \usepackage{mycolor}      % use mycolor.sty
  \begin{document}
  Rest content of the article.
  \end{document}
\end{minted}

3、当文档编辑完成,定稿之后,再修改 mycolor.sty 中的颜色定义,将其变为白底黑字。
当然,也可以直接将 mycolor.sty 中的内容注释掉,以达到恢复默认色的效果。

\section{\LaTeX文件如何拆分并进行独立编译}

对于较大的LaTeX文件,需要将其分析成多个 .tex 文件。比如编辑一本书的时候可以将各章独立为chapter1.tex、chapter2.tex、chapter3.tex ……,然后在主文件 main.tex 中包含进来。如:
\begin{minted}[frame=single]{tex}
  \documentclass{book}
  \begin{document}
  \title{A LaTeX Book}
  \author{}
  \date{}
  \maketitle
  \include{chapter1}
  \include{chapter2}
  \include{chapter3}
  % include other chapters
  \end{document}
\end{minted}

分析后会引入一个新问题:如何让分析后的 chapN.tex 文件即能够被 main.tex 引用,又能够单独编译?
毕竟,在编写 chapterN.tex 文件时,能单独编译可以直接查看内容是否正确。

解决的办法是:条件编译。即:在 main.tex 中定义一个特殊的值;在 chapterN 中根据是否有该定义写不同的代码。

main.tex 中定义一个特殊值: allfiles。

\begin{minted}[frame=single]{tex}
  \documentclass{book}
  \def\allfiles{}   % define 'allfiles'
  \begin{document}
  % the rest of content.
  \end{document}
\end{minted}

在 chapterN 中判断有无 allfiles 定义。

\begin{minted}[frame=single]{tex}
  \ifx\allfiles\undefined
    \documentclass{article}
    \begin{document}
    \title{Title of the chapter}
    % more command for title & toc.
  \else
    \chapter{Title of the chapter}
  \fi

  %
  % content of the chapter.
  %

  \ifx\allfiles\undefined
    \end{document}
  \if
\end{minted}

\section{\LaTeX工程}

当要写内容较多时,除了需要将一个 .tex 分拆为多个 .tex 文件,
还可能涉及到文档中引用的外部图片、参考资料等相关信息,
有必要将这些内容归类存放,方便维护。

我所使用的方法:

\begin{enumerate}
  \item 在工程根目录下,创建一个 main.tex ,只负责引用其它 .tex 文件。
  \item 创建一个 chapter 子目录,将各章节的 .tex 文件放在该目录下。
  \item 创建一个 picture 子目录,将需要引用的图片统一存放在该目录下。
\end{enumerate}

% 重复判断是否是从 main.tex 开始编译 
\ifx\allfiles\undefined
  \end{document}
\fi
